\pagestyle{empty}

\noindent \textbf{\Large CD Használati útmutató}

\vskip 1cm

\begin{itemize}
	\item programok/
		\begin{itemize}
			\item Ebben a mappában találhatóak a felhasznált kódok illetve a méréseket tartalmazó táblázatok.
			\item README.txt fájl leírja, hogy az egyes könyvtárakban lévő programok mire szolgálnak és mi található bennük.
		\end{itemize}
	\item szakdolgozat/
			\begin{itemize}
				\item \textbf{dolgozat.pdf}
				\item Itt találhatóak meg a dolgozat forrás fájljai.
				\item README.txt fájl segít eligazodni részletesen a fájlok között.
		\end{itemize}
		
	\item UTMUTATO.pdf részletesebb útmutatás a CD használatához.
\end{itemize}

%Ennek a címe lehet például \textit{A mellékelt CD tartalma} vagy \textit{Adathordozó használati útmutató} is.

%Ez jellemzően csak egy fél-egy oldalas leírás.
%Arra szolgál, hogy ha valaki kézhez kapja a szakdolgozathoz tartozó CD-t, akkor tudja, hogy mi hol van rajta.
%Jellemzően elég csak felsorolni, hogy milyen jegyzékek vannak, és azokban mi található.
%Az elkészített programok telepítéséhez, futtatásához tartozó instrukciók kerülhetnek ide.
%
%A CD lemezre mindenképpen rá kell tenni
%\begin{itemize}
%\item a dolgozatot egy \texttt{dolgozat.pdf} fájl formájában,
%\item a LaTeX forráskódját a dolgozatnak,
%\item az elkészített programot, fontosabb futási eredményeket (például ha kép a kimenet),
%\item egy útmutatót a CD használatához (ami lehet ez a fejezet külön PDF-be vagy MarkDown fájlként kimentve).
%\end{itemize}
