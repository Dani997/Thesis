\Chapter{Bevezetés}

Az adatbázismotorok működésének optimalizálásához különféle módszerek állnak rendelkezésre. Ezek többségében egy-egy adott szoftverhez megfelelőek, általánosan alkalmazható módszerből kevés van.

Az adatbázis rendszerek képesek megmutatni, miként fognak végrehajtani egy-egy lekérdezést, így ezt felhasználva lehetőségünk van olyan módosításokat végrehajtani amelyekkel javíthatjuk a teljesítményt. Sajnos bizonyos esetekben ez olyan költségekkel járhat, mint az adatbázis teljes újratervezése.

A hatékonyság növelésének egyik megoldása lehet egy saját kliens alkalmazás elkészítése. Ezzel olyan módon szeretnénk csökkenti a lekérdezések sebességét, hogy a számításigényes műveleteket nem az adatbázis szerver, hanem a kliens gép erőforrásaival végeztetjük el. A dolgozatban ez a megközelítés kerül bemutatásra.

Ekkor kerül elő az a felvetés, hogy ki lehetne használni a grafikus kártya köztudottan magas számítási teljesítményét és párhuzamosítási képességeit. Ez elsősorban a több processzormagos architektúrájából adódik. Jelenleg is használják ezeket az eszközöket különféle számítások elvégzésére, például kripto pénzek bányászatakor, amikor úgy nevezett \textit{hash}-eket állítanak elő. Egyes esetekben a grafikus kártyák ennél a folyamatnál akár 800x is gyorsabbak lehetnek mint az azonos számításokat végző, egy CPU-t (\textit{Central Processing Unit}) használó változat \cite{crypto}. Az előbbiek alapján jogosan merül fel a kérdés, hogyan lehet SQL lekérdezéseket megvalósítani GPU (\textit{Graphics Processing Unit}) segítségével, illetve milyen előnyökkel és hátrányokkal járhat ez a megoldás.

A dolgozat a MySQL adatbázismotor esetében mutatja be az optimalizálás lehetőségeit, az SQL lekérdezések végrehajtásának elemzési módjait. Áttekintés szintjén kitér az OpenCL nyelvre, majd egy saját fejlesztésű, a MySQL-hez már egyébként rendelkezésre álló \textit{driver} programra épülő hatékonyabb elérési módot biztosító szoftvert mutat be. A dolgozatban a számítási teljesítmény vizsgálatához számos mérés és annak eredményei is részletezésre kerülnek.




%A videokártyák számítási számítási teljesítménye köztudottan rendkívül magas mégsem terjed el és váltotta fel a CPU -kat.

%Ezek az eszközök többször annyi számítási maggal rendelkeznek mint egy átlagos processzor ezért tudnak olyan gyorsak lenni a párhuzamos, grafikai számítások során. A kripto pénzek bányászását is ilyen kártyákkal végzik. Ennél a feladatnál úgy nevezett Hesh-eket állítanak elő. Egyes esetekben a grafikus kártyák akár 800x is gyorsabbak lehetnek mint egy CPU.

%Felmerül tehát a kérdés, milyen egyéb számítások során lehetne még kihasználni az eszköz nagy mértékű párhuzamosíthatóságát.



