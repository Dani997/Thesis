\Chapter{Bevezetés}

Napjainkban viszonylag kevés módszer módszer áll rendelkezésünkre, ha egy adatbázis működését optimalizálni szeretnénk.

Az adatbázis rendszerek képesek megmutatni, miként fognak végrehajtani egy-egy lekérdezést, így ezt felhasználva lehetőségünk van olyan módosításokat végrehajtani amelyekkel javíthatjuk a teljesítményt. De sajnos bizonyos esetekben ez olyan költségekkel járhat, mint az adatbázis teljes újratervezése.

A hatékonyság növelésének egyik megoldása lehet egy saját kliens alkalmazás elkészítése. Ezzel olyan módon szeretnénk csökkenti a lekérdezések sebességét, hogy a számítás igényes műveleteket, nem az adatbázis szerver, hanem a kliens gép erőforrásaival végeztetjük el. Ekkor kerül elő az a felvetés, hogy ki lehetne használni a grafikuskártya köztudottan magas számítási teljesítményét.

Ennek a kiemelkedő feldolgozó képessége a sok processzormagból és az ebből származó párhuzamosságból adódik. Jelenleg is használják ezeket az eszközöket különféle számítások elvégzésére, például kripto pénzek bányászatakor, amikor úgy nevezett hesh -eket állítanak elő. Egyes esetekben a grafikus kártyák ennél a folyamatnál akár 800x is gyorsabbak lehetnek mint egy CPU (\textit{Central Processing Unit}). \cite{crypto}

Tehát jogosan merül fel a kérdés, hogyan lehet SQL lekérdezéseket megvalósítani GPU (\textit{Graphics Processing Unit}) -n, illetve milyen előnyökkel és hátrányokkal járhat ez a megoldás.






%A videokártyák számítási számítási teljesítménye köztudottan rendkívül magas mégsem terjed el és váltotta fel a CPU -kat.

%Ezek az eszközök többször annyi számítási maggal rendelkeznek mint egy átlagos processzor ezért tudnak olyan gyorsak lenni a párhuzamos, grafikai számítások során. A kripto pénzek bányászását is ilyen kártyákkal végzik. Ennél a feladatnál úgy nevezett Hesh-eket állítanak elő. Egyes esetekben a grafikus kártyák akár 800x is gyorsabbak lehetnek mint egy CPU.

%Felmerül tehát a kérdés, milyen egyéb számítások során lehetne még kihasználni az eszköz nagy mértékű párhuzamosíthatóságát.



