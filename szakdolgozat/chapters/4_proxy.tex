\Chapter{Saját kód elhelyezése}

\Section{A MySQL adatbázis fordítása és telepítése}

A MySQL forráskódból való fordítását egy frissen telepített és naprakészre frissített Manjaro 20.2.1 es linux disztribúción végeztem el. Telepítés során non-free grafikus drivert választottam.
A forrásállomány bárki számára elérhető, és a következő paranccsal klónozható:
\begin{python}
 git clone https://github.com/mysql/mysql-server.git
\end{python}
A fordítás sikerességéhez rendszerenként eltérő csomagok telepítésére lehet szükség. Jelen esetben a következők telepítéseket igényelte a folyamat.
% QUEST: Miért snap-es csomaggal lett telepítve?
% ANSWER: A cmake telepítése a gyártó hivatalos weboldalán található Manjato rendszerhez készült segédlet alapján lett elvégezve.
\begin{python}
 $ sudo snap install cmake --classic
 $ sudo pacman -S rpcsvc-proto
 $ sudo pacman -S pkgconfig
 $ sudo pacman -S make
 $ sudo pacman -S gcc
 $ sudo pacman -S bison
\end{python}%$
%A \textit{Boost} függvénykönyvtár később kapcsoló használatával letölthető.
A következő lépés a CMake futtatása. Ehhez célszerű létrehozni egy mappát ahová az új állományok létrejöhetnek. Ezt nem kötelező megtenni, a forrás könyvtárban is ki lehet adni a parancsot, ehhez a \texttt{CMake} megfogja adni a szükséges kapcsolót illetve figyelmeztet arra, hogy nem ajánlatos.
\begin{python}
 mkdir build
\end{python}
Amint ez kész a forrás mappájában futtathatjuk a következő parancsot:
\begin{python}
 cmake ./ ../build -DDOWNLOAD_BOOST=1 -DWITH_BOOST=../boost/
\end{python}
A megadott kapcsolóval a boost könyvtár letöltődik a megadott helyre és a \texttt{CMake} megjegyez ezt. Ha a generátor hibába ütközik a szükséges csomagokat telepíteni kell és a \textit{CMakeCache.txt} fájlt törtölni. Ezek után a \texttt{CMake} újra futtatható, de már a \textit{Boost} -hoz tartozó kapcsolók nélkül.

A sikeres futás jelzi, hogy minden készen áll a fordításra, nincs hiányzó csomag.
A \texttt{make} parancs futtatása következik. Ez egy hosszú folyamat, jelen konfiguráción 120 percet vett igénybe. 

A sikeres fordítást követően létre kell hoznunk egy data mappát a szerver számára.
\begin{python}
 mkdir data
\end{python}
Első futtatást rendszergazdaként a megadott kapcsolóval kell végrehajtani, csak ekkor tudja a szerver létrehozni a fájljait!
\begin{python}
 sudo ./bin/mysqld --initialize
\end{python}
Ekkor kapunk egy jelszót az a szerverre való csatlakozáshoz. Következő futtatás előtt szükség lehet a data mappa jogosultságainak állítására, különben a szerver a normál indítás során nem tudja inicializálni az InnoDB-t olvasási hiba miatt.
\begin{python}
 sudo chmod -R 777 ./data
\end{python}
Ezután futtathatjuk a szervert
\texttt{./bin/mysqld}
Vagy telepíthetjük is azt a \texttt{make install} paranccsal, de ez jelen esetben szükségtelen.



Csatlakozás a szerverre:
\begin{python}
 ./bin/mysql -u root -p
\end{python}
A generált jelszó bemásolásával belépünk. Majd megváltoztatjuk a jelszót.
\begin{python}
ALTER USER 'root'@'localhost' IDENTIFIED BY 'uj jelszo';
\end{python}
Vagy használhatjuk a MySQL Workbench-et ami első belépéskor megkér minket a módosításra.
% TODO: SQL-es környezetet lesz majd célszerű használni helyette!


\Section{A MySQL Connector}

A \texttt{Connector}, \texttt{MySQL} alapszabványú drivereket biztosít különböző nyelveken, amelyek lehetővé teszik a fejlesztők számára, hogy adatbázis alkalmazásokat írjanak a támogatott nyelveken. Ezen kívül egy natív \texttt{C} könyvtár teszi lehetővé azt, hogy a \texttt{MySQL} -t közvetlenül beágyazhassák alkalmazásaikba. \newline
A következő \texttt{MySQL} által fejlesztett driverek érhetőek el: \newline
\texttt{ADO.NET}, \texttt{ODBC}, \texttt{JDBC}, \texttt{Node.js}, \texttt{Python}, \texttt{C++}, \texttt{C} és \texttt{C API} a klienshez. \newline
Az alábbiakat pedig a \texttt{MySQL} közössége fejleszti:\newline
\texttt{PHP}, \texttt{Perl}, \texttt{Ruby}, \texttt{C+ Wrapper}.

*https://www.mysql.com/products/connector/

\SubSection{A Connector/C++ használata}

Az állományok letölthetőek a készítő hivatalos weboldaláról:\newline
https://dev.mysql.com/downloads/connector/cpp/
\begin{itemize}
	\item Linux - Generic
	\item All
	\item Linux - Generic (glibc 2.12) (x86, 64-bit), Compressed TAR Archive
\end{itemize}
Letöltés után kicsomagoljuk. A tartalma egy \texttt{include} és egy \texttt{lib64} mappa. Az \texttt{include} mappa tartalmát a \texttt{usr/include} könyvtárba másoljuk, a \texttt{lib64} -ét pedig az \texttt{usr/lib64} mappába.

Ezután szükség lesz egy külön felhasználóra amellyel a program csatlakozni fog a szerverünkre. Ezt létrehozhatjuk a \texttt{Workbench} alkalmazásban is.
\begin{python}
	CREATE USER 'newuser'@'localhost' IDENTIFIED BY 'password';
\end{python}

\SubSection{Első program}

Hozzunk létre egy \texttt{connectortest.cpp} nevű fájlt.
Szükséges osztályok létrehozása:
\begin{cpp}
	sql::Driver *driver;
	sql::Connection *con;
	sql::Statement *stmt;
	sql::ResultSet *res;
	sql::PreparedStatement *pstmt;
\end{cpp}
Csatlakozás a szerverhez és adatbázis kiválasztása:
\begin{cpp}
	driver = get_driver_instance();
	con = driver->connect("tcp://192.168.0.43:3306", "program", "a");
	con->setSchema("thesis");
\end{cpp}
A következő két sor maga a lekérdezés. Az első utasítás elkészíti a szövegből a szerver számára is értelmezhető utasítást, a második sor pedig elküldi azt a szervernek és megvárja a választ:
\begin{cpp}
	pstmt = con->prepareStatement("SELECT * FROM T3");
	res = pstmt->executeQuery();
\end{cpp}

Az eredményt a \texttt{res} objektumon végig lépkedve tudjuk kiolvasni, a lekért oszlop nevének megadásával és a típusnak megfelelő \texttt{get} függvénynél.
\begin{cpp}
	while (res->next())
	cout << res->getInt("c1p3") << "\t" << res->getInt("c2") << "\t"
	<< res->getInt("c3") << "\t" << res->getInt("c4") << "\t"
	<< res->getInt("fk_p1_p3") << "\t" << res->getInt("fk_p2_p3") 
	<< endl;
\end{cpp}
Lekérdezés után fontos törölni az eredményhalmazt és az elkészített utasítást, ugyanis ezek memóriát foglalnak, és több lekérdezés esetén megtelhet akár a teljes memória is.
\begin{cpp}
	delete res;
	delete pstmt;
\end{cpp}
Futtatáshoz és fordításhoz a következő parancsokat használhatjuk: 
\begin{python}
	g++ -D_GLIBCXX_USE_CXX11_ABI=0 connectortest.cpp -o connectortest.out 
	-lmysqlcppconn
	
	./connectortest.out
\end{python}
