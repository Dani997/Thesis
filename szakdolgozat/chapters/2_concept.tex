\Chapter{A MySQL telepítése forráskódból}


A MySQL forráskódból való fordítását egy frissen telepített és naprakészre frissített Manjaro 20.2.1 es linux disztribúción végeztem el. Telepítés során non-free grafikus drivert választottam.
A forrásállomány bárki számára elérhető, és a következő paranccsal klónozható:
\begin{python}
 git clone https://github.com/mysql/mysql-server.git
\end{python}
A fordítás sikerességéhez rendszerenként eltérő csomagok telepítésére lehet szükség. Jelen esetben a következők telepítéseket igényelte a folyamat.
% QUEST: Miért snap-es csomaggal lett telepítve?
% ANSWER: A cmake telepítése a gyártó hivatalos weboldalán található Manjato rendszerhez készült segédlet alapján lett elvégezve.
\begin{python}
 $ sudo snap install cmake --classic
 $ sudo pacman -S rpcsvc-proto
 $ sudo pacman -S pkgconfig
 $ sudo pacman -S make
 $ sudo pacman -S gcc
 $ sudo pacman -S bison
\end{python}%$
%A \textit{Boost} függvénykönyvtár később kapcsoló használatával letölthető.
A következő lépés a CMake futtatása. Ehhez célszerű létrehozni egy mappát ahová az új állományok létrejöhetnek. Ezt nem kötelező megtenni, a forrás könyvtárban is ki lehet adni a parancsot, ehhez a \texttt{CMake} megfogja adni a szükséges kapcsolót illetve figyelmeztet arra, hogy nem ajánlatos.
\begin{python}
 mkdir build
\end{python}
Amint ez kész a forrás mappájában futtathatjuk a következő parancsot:
\begin{python}
 cmake ./ ../build -DDOWNLOAD_BOOST=1 -DWITH_BOOST=../boost/
\end{python}
A megadott kapcsolóval a boost könyvtár letöltődik a megadott helyre és a \texttt{CMake} megjegyez ezt. Ha a generátor hibába ütközik a szükséges csomagokat telepíteni kell és a \textit{CMakeCache.txt} fájlt törtölni. Ezek után a \texttt{CMake} újra futtatható, de már a \textit{Boost} -hoz tartozó kapcsolók nélkül.

A sikeres futás jelzi, hogy minden készen áll a fordításra, nincs hiányzó csomag.
A \texttt{make} parancs futtatása következik. Ez egy hosszú folyamat, jelen konfiguráción 120 percet vett igénybe. 

A sikeres fordítást követően létre kell hoznunk egy data mappát a szerver számára.
\begin{python}
 mkdir data
\end{python}
Első futtatást rendszergazdaként a megadott kapcsolóval kell végrehajtani, csak ekkor tudja a szerver létrehozni a fájljait!
\begin{python}
 sudo ./bin/mysqld --initialize
\end{python}
Ekkor kapunk egy jelszót az a szerverre való csatlakozáshoz. Következő futtatás előtt szükség lehet a data mappa jogosultságainak állítására, különben a szerver a normál indítás során nem tudja inicializálni az InnoDB-t olvasási hiba miatt.
\begin{python}
 sudo chmod -R 777 ./data
\end{python}
Ezután futtathatjuk a szervert
\texttt{./bin/mysqld}
Vagy telepíthetjük is azt a \texttt{make install} paranccsal, de ez jelen esetben szükségtelen.



Csatlakozás a szerverre:
\begin{python}
 ./bin/mysql -u root -p
\end{python}
A generált jelszó bemásolásával belépünk. Majd megváltoztatjuk a jelszót.
\begin{python}
ALTER USER 'root'@'localhost' IDENTIFIED BY 'uj jelszo';
\end{python}
Vagy használhatjuk a MySQL Workbench-et ami első belépéskor megkér minket a módosításra.
% TODO: SQL-es környezetet lesz majd célszerű használni helyette!
