\Chapter{Adatbázisok teljesítményének optimalizálása}

% TODO: Indexelés kifejtése

\Section{Indexelés}
%https://www.sqlshack.com/query-optimization-techniques-in-sql-server-the-basics/
Egy táblázat adataihoz kétféleképpen lehet hozzáférni, úgynevezett szkenneléssel illetve kereséssel. A keresés nem más, mint mikor valamilyen szűrő feltétel alapján választunk ki rekordokat. Ezek a szűrőfeltételek általában keskeny szűrők, azaz sok adatból kevés értékre igazak. A szkennelés, mikor egy teljes indexet keresünk a megfelelő értékek visszaadásához. Egy tábla akár több millió sor is lehet, kereséskor ennek akár minden a feltételben szereplő értékét vizsgálni kellhet. Ugyan ezen a táblán sokkal gyorsabban is át lehet haladni ha az indexek bináris fáját vizsgáljuk, ilyenkor anélkül adható vissza a végeredmény, hogy minden adatot át kellene vizsgálni.
Néhány felmerülő gondolat amit optimalizáláskor figyelni kell:
\begin{itemize}
\item Van -e olyan index amely használható a lekérdezéskor?
\item Ha nincs indexelés, akkor hozzunk létre?
\item Elég gyakori az adott lekérdezés ahhoz, hogy ez megérje?
\begin{itemize}
\item Előnye, hogy gyorsítja a lekérdezést.
\item Hátránya, hogy csökkenti az írási sebességet.
\end{itemize}
\item Érvényes a szűrő? Szoktak szűrni az adott oszlop alapján?
\item Elkerülhető a vizsgálat a lekérdezésnél? Néhány lekérdezés szinte mindent magába foglal, ezért teljes táblavizsgálatot igényel.
\end{itemize}



% TODO: Séma normalizálás/denormalizálás

%https://docs.microsoft.com/en-us/office/troubleshoot/access/database-normalization-description

\Section{Normalizálás és denormalizálás}

\SubSection{Normalizálás}
A normalizálás az adatok adatbázisba szervezésének folyamata. Ez magába foglalja a táblák létrehozását és az azok közötti kapcsolatok kialakítását. Mindezt olyan szabályok szerint melyek segítenek az adatok védelmében, az adatbázis rugalmassá tételében, a redundancia elkerülésében és következetlen függőségek kiküszöbölésében.

A redundancia pazarolja a lemezterületet és karbantartási problémákat okoz. Egyazon adat több helyen való tárolása igényli, hogy mindenhol ugyan úgy módosuljon az adat, különben anomáliák jöhetnek létre.

A következetlen függőség a táblák nem megfelelően felépítéséből adódik. Bizonyos adatelemek ha nem megfelelő táblába kerülnek, akkor nehezítik annak elérését, illetve teljesen megszakadhat az adatok megtalálásának útja.

Ezek kiküszöbölésére léteznek szabályok, melyeket normál formáknak nevezünk. Ha egy séma teljesíti az első normál forma előírásait, akkor azt mondjak első normálformában van. A harmadik normálforma tekinthető a legtöbb alkalmazáshoz szükséges legmagasabb szintnek.

°%https://tudasbazis.sulinet.hu/hu/szakkepzes/informatika/adatbazis-kezeles/a-normalizalas-folyamata/normal-formak



% TODO: Egyéb paraméterek, beállításaik
