\Chapter{Adatbázisok teljesítményének optimalizálása}

% TODO: Indexelés kifejtése

\Section{Indexelés}
%https://www.sqlshack.com/query-optimization-techniques-in-sql-server-the-basics/
Egy táblázat adataihoz kétféleképpen lehet hozzáférni, úgynevezett szkenneléssel illetve kereséssel. A keresés nem más, mint mikor valamilyen szűrő feltétel alapján választunk ki rekordokat. Ezek a szűrőfeltételek általában keskeny szűrők, azaz sok adatból kevés értékre igazak. A szkennelés, mikor egy teljes indexet keresünk a megfelelő értékek visszaadásához. Egy tábla akár több millió sorból is állhat, kereséskor ennek minden sorát át kell vizsgálni. Ugyan ezen a táblán sokkal gyorsabban is át lehet haladni ha az indexek bináris fáját vizsgáljuk, ilyenkor anélkül adható vissza az végeredmény, hogy minden rekordot át kellene vizsgálni.
Néhány felmerülő gondolat amit optimalizáláskor figyelni kell: \cite{index}
\begin{itemize}
\item Van -e olyan index amely használható a lekérdezéskor?
\item Ha nincs indexelés, akkor hozzunk létre?
\item Elég gyakori az adott lekérdezés ahhoz, hogy ez megérje?
\begin{itemize}
\item Előnye, hogy gyorsítja a lekérdezést.
\item Hátránya, hogy csökkenti az írási sebességet.
\end{itemize}
\item Érvényes a szűrő? Szoktak szűrni az adott oszlop alapján?
\item Elkerülhető a vizsgálat a lekérdezésnél? Néhány lekérdezés szinte mindent magába foglal, ezért teljes táblavizsgálatot igényel.
\end{itemize}



% TODO: Séma normalizálás/denormalizálás

%https://docs.microsoft.com/en-us/office/troubleshoot/access/database-normalization-description

\Section{Normalizálás és denormalizálás}

\SubSection{Normalizálás}
A normalizálás az adatok adatbázisba szervezésének folyamata. Ez magába foglalja a táblák létrehozását és az azok közötti kapcsolatok kialakítását. Mindezt olyan szabályok szerint melyek segítenek az adatok védelmében, az adatbázis rugalmassá tételében, a redundancia elkerülésében és a következetlen függőségek kiküszöbölésében.

A redundancia pazarolja a lemezterületet és karbantartási problémákat okoz. Egyazon adat több helyen való tárolása igényli, hogy mindenhol ugyan úgy módosuljon az adat, különben anomáliák jönnek létre.

A következetlen függőség a táblák nem megfelelően felépítéséből adódik. Bizonyos adatelemek ha nem megfelelő táblába kerülnek, akkor nehezítik annak elérését, illetve teljesen megszakadhat az adatok megtalálásának útja.

Ezek kiküszöbölésére léteznek szabályok, melyeket normál formáknak nevezünk. Ha egy séma teljesíti az első normál forma előírásait, akkor azt mondjak első normálformában van. A harmadik normálforma tekinthető a legtöbb alkalmazáshoz szükséges legmagasabb szintnek, bár öt van. \cite{normaliz}

\SubSection{Denormalizálás}
%https://web.archive.org/web/20171201030308/https://pdfs.semanticscholar.org/2c79/069c01ba8d598f32e61fe367ef6d261a0cb4.pdf
Általános vélekedés, hogy a normalizáció rontja a válaszidőt. A teljes normalizálás logikailag különálló entitásokat eredményez, melyek fizikailag is különállóan vannak eltárolva. Ennek hatása lehet, hogy a feldolgozás több erőforrást igényel.
A denormalizálás célja az olvasási sebesség javítása azzal, hogy szándékosan idéznek elő redundanciát vagy csoportosítanak adatokat. \cite{denormaliz}
\begin{itemize}
\item A folyamat magas szakértelmet igényel.
\item Figyelembe kell venni az adatbázis logikai és fizikai felépítését is.
\item Csak már bizonyos szintig normalizált sémákon végezhető el.
\item Főként sok lekérdezés, és kevés frissítés esetén érdemes alkalmazni.
\end{itemize}



% TODO: Egyéb paraméterek, beállításaik

\SubSection{Egyéb optimalizálási paraméterek}

Az adatbázis sebességét nem csak annak megfelelő kialakításával segíthetjük elő, hanem a megfelelő adatbázis kezelő kiválasztásával és annak helyes konfigurálásával is.
Jelen esetben a \texttt{MySQL}-t használjuk, melyhez több motor is elérhető. Alapértelmezett módon ez az \texttt{InnoDB}, de ezen kívül elérhető a \texttt{MyISAM}-is. Utóbbinak hátránya, hogy nem támogatja a sor szintű zárolást, idegen kulcsokat és tranzakciókat. A \texttt{MyISAM} ezek miatt viszonylag ritkán jobb döntés. Íme pár fontosabb konfigurációs beállítás: \cite{other_optimization}
\begin{itemize}
\item \texttt{innodb buffer pool size}: A motor gyorsítótárának mérete. Elérheti a szerver memória 80-90\% -át is ha ez szükséges.
\item \texttt{innodb io capacity}: Megadja a \texttt{MySQL}-nek, hogy hány I/O műveletet végezhet. Ezzel háttértárnak megfelelő korlát adható, így nem fog a szerver  egyszerre túl sok ilyen műveletet végezni.
\item \texttt{query cache limit} és \texttt{query cache size}: Túl nagy terhelés esetén a gyorsítótár használhatatlan, ilyenkor érdemes kikapcsolni, hogy megspóroljuk annak kezelési költségét.
\item \texttt{slow query log, long query time}: Megmutatja mely lekérdezések lassabbak a megadott időnél. Erről .log fájl készül amely megmutatja, hány sort érint a lekérdezés, mikor és ki hajtotta végre.
\end{itemize}
