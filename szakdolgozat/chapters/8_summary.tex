\Chapter{Összefoglalás}

A dolgozatban sok érdekes információ kiderült a MySQL lekérdezések és az OpenCL-es megvalósítások hatékonyságával kapcsolatban.

A MySQL Workbench által előállított grafikus végrehajtási terv hatékonyan nyújt segítséget a lekérdezéseink optimalizálásához. Használatával egyszerű módon találhatjuk adatbázisunk hiányosságait, legyen az a felépítéséből vagy akár csak egy indexelés hiányából adódó hátrány. Ez az eszköz bárki számára könnyedén elérhető és sok esetben nem igényel különleges szaktudást sem. Ideális esetben akár egy indexelés hozzáadásával is töredékére csökkenthetjük lekérdezéseink idejét.

Áttérve a lekérdezések OpenCL-es megvalósítására, azt a következtetést vontam le, hogy vannak olyan esetek, amikor sebesség szempontjából jelentős előnyre lehet szert tenni, ha kihasználjuk a grafikus kártyák teljesítményét. De ahhoz, hogy ez a gyakorlatban is megvalósítható legyen nagyon sok előzetes vizsgálatra van szükség. A hatékonyság függ:
\begin{itemize}
\item az adatbázis szerver sebességétől,
\item a kliensgép erőforrásaitól,
\item a lekérdezés bonyolultságától,
\item a séma felépítésétől.
\end{itemize}

Erre megoldást nyújthat egy a proxy modell-el felépített kliens alkalmazás. Ennél dönthet a felhasználó arról, hogy bekapcsolja az \texttt{OpenCL}-es feldolgozót, vagy rábízhatja a programra ezt a döntést. A kiértékelőt egyedileg be kell paraméterezni adott sémához. A döntéshez többek között szükség van a a táblák felépítésre és arra, mely oszlopok indexeltek. További tényezőket is érdemes lehet figyelembe venni, mint például egy oszlop lehetséges értéktartománya vagy akár az adatok várható eloszlása, mellyel becsülhető egy szűrő szélessége. 

Egy ilyen kliensalkalmazás megírása rendkívül bonyolult és időigényes ezért nagyon kevés olyan eset lehet, ahol ez a befektetett idő és energia valóban megtérül. 
