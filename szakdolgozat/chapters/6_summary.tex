\Chapter{Összefoglalás}

A dolgozatban sok érdekes információ kiderült a mysql lekérdezések és az opencl -es megvalósítások hatékonyságával kapcsolatban.

A MySQL Workbench által előállított grafikus végrehajtási terv hatékonyan nyújt segítséget a lekérdezéseink optimalizálásához. Használatával egyszerű módon találhatjuk adatbázisunk hiányosságait, legyen az a felépítéséből vagy akár csak egy indexelés hiányából adódó hátrány. Ez az eszköz bárki számára könnyedén elérhető és sok esetben nem igényel különleges szaktudást sem. Ideális esetben akár egy indexelés hozzáadásával is töredékére csökkenthetjük lekérdezéseink idejét.

Áttérve a lekérdezések OpenCL -es megvalósítására, azt a következtetést vontam le, hogy vannak olyan esetek, amikor sebesség szempontjából jelentős előnyre lehet szertetenni, ha kihasználjuk a grafikus kártyák teljesítményét. De ahhoz, hogy ez a gyakorlatban is megvalósítható legyen nagyon sok előzetes vizsgálatra van szükség. A hatékonyság függ:
\begin{itemize}
\item Az adatbázis szerver sebességétől.
\item A kliensgép erőforrásaitól.
\item A lekérdezés bonyolultságától.
\end{itemize}
Hogy valóban használható legyen a módszer sok dolognak kellene teljesülnie.
Például, hogy a lekérdezések idejére az adatbázis tartalma perzisztens legyen, ugyan is a kártya puffereiben lévő adat elavulttá válhat, melynek frissítése erősen rombolja a hatékonyságot. 

Egy ilyen kliensalkalmazás megírása rendkívül bonyolult és időigényes ezért nagyon kevés olyan eset lehet, ahol ez a befektetett idő és energia valóban megtérül. 






