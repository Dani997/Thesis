\Chapter{Bevezetés}

A videokártyák számítási számítási teljesítménye köztudottan rendkívül magas mégsem terjed el és váltotta fel a CPU -kat.

Ezek az eszközök többször annyi számítási maggal rendelkeznek mint egy átlagos processzor ezért tudnak olyan gyorsak lenni a párhuzamos, grafikai számítások során. A kripto pénzek bányászását is ilyen kártyákkal végzik. Ennél a feladatnál úgy nevezett Hesh-eket állítanak elő. Egyes esetekben a grafikus kártyák akár 800x is gyorsabbak lehetnek mint egy CPU.

Felmerül tehát a kérdés, milyen egyéb számítások során lehetne még kihasználni az eszköz nagy mértékű párhuzamosíthatóságát.

https://bitemycoin.com/cryptocurrency-mining/whats-the-difference-between-cpu-and-gpu-mining/

Jelen dolgozat azt vizsgálja, hogy miképpen lehet az SQL lekérdezéseket párhuzamosítani, milyen költségekkel jár az adatok eljuttatása a kártyába és milyen hatékonysággal lehet őket ott feldolgozni. 